\chapter*{Resumen}

La robótica está sufriendo una rápida y enorme evolución en los últimos años, extendiéndose desde sus orígenes en entornos estructurados, seguros y simples en las industrias hacia ambientes complejos y no estructurados. Esta expansión se debe principalmente a dos factores, el abaratamiento de los componentes y su mejora. A esto también hay que sumarle la evolución del software en general y más específicamente al desarrollo de nuevos algoritmos y técnicas de aprendizaje automático que dotan a los robots de la capacidad de realizar tareas que hasta hace unos años parecían impensables, como los vehículos con conducción autónoma.

Estos desarrollos han causado que tanto el número como la complejidad de las aplicaciones robóticas haya aumentado de manera exponencial. Esto ha conllevado la creación de distintos tipos de software con el fin de abstraer y simplificar su desarrollo, como por ejemplo el surgimiento de \textit{middlewares} robóticos que ofrecen una capa de abstracción y de estandarización de los componentes que formarán parte de las aplicaciones así como herramientas para su desarrollo. Continuando ese objetivo han aparecido varias soluciones con el fin de reducir la complejidad de la creación de aplicaciones robóticas, así como un desarrollo más rápido y eficiente, siendo un ejemplo de esto los IDE (\textit{Integrated Developer Enviroment} o Entorno de Desarrollo Integrado) robóticos (como Asimovo o Flowstate, de Intrinsic Google).

En este trabajo se presenta la mejora de la herramienta BT Studio, un IDE web para la programación de aplicaciones robóticas basadas en \textit{árboles de comportamiento}, un paradigma de programación en auge en la industria robótica. En esta herramienta, los usuarios tienen la capacidad de programar aplicaciones robóticas completamente desde el navegador mediante acciones escritas en Python en el editor incluido y árboles de comportamiento que son editados en un editor visual basado en bloques. Posteriormente, los usuarios pueden descargar sus aplicaciones para ejecutarlas en local o usar el entorno integrado con el visualizador en la propia página web.

\vspace{.5cm}

\textbf{Palabras clave:} robótica, árboles de comportamiento, inteligencia artificial, frontend, backend, ROS 2, Docker. 